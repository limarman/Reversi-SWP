\section{Task 2}
Wir haben die folgenden Karten entworfen:

\subsection{2-Spieler-Karte}
Die zwei-Spieler-Karte wird von uns auch liebevoll \textit{Transversi} genannt. Sie ist eine Abwandlung der Standard Reversi Karte, wobei alle Felder ausschließlich über Transitionen verbunden sind. Die Motivation dahinter ist die, dass wir vor der Implementation von \textit{Blöcken} bei der Berechnung der Mobilität keine Transitionen beachtet haben. Wir hoffen, dass andere Gruppen dies auch nicht machen und wir mit unseren \textit{Blöcken} so einen Vorteil auf dieser Karte erhalten.

\subsection{4-Spieler-Karte}
Diese Karte trägt den Spitznamen \textit{Laughing Trophy}. Wir hoffen  auf ein paar interessante Matches auf dieser Karte. Wir in dieser Karte ein paar Fallen eingebaut: Es gibt ein unerreichbares Inversion Feld. Dies ist eine Antwort auf die \textit{erreichbaren Komponenten}, an denen wir arbeiten. Wir vermuten, dass andere Gruppen ermitteln, wie viele Inversionstones es gibt und entsprechend ihr verhalten irgendwie anpassen. Mit einem unerreichbaren Stein erhoffen wir uns, dass andere Gruppen nicht feststellen, dass dieser niemals ausgeführt wird und deshalb schlechter spielen, so wie wir, ohne die \textit{erreichbaren Komponenten}. Die Bonus Felder im oberen Teil der Karte dienen auch dem Ziel, andere Gruppen aus dem Konzept zu bringen.

\subsection{8-Spieler-Karte}
Diese Karte ist relativ klein für eine 8-Spieler Karte und hat die besondere Eigenschaft, dass durch die verstreuten Holes die Mobilität eingeschränkt wird und sonst aufgrund der wenigen Sonderfelder das Spiel nicht forcierend ist. Wir erhoffen uns, dass unsere Heuristik damit den anderen Gruppen überlegen ist. Schaffen wir es durch einen Mobilitätsvorteil eine gewisse Oberhand zu gewinnen, können wir im späteren und konkreteren Verlauf des Spiels dies ausnutzen und weitere Override-Steine krallen. Bonus-Felder gibt es nur im unteren Bereich der Map, wohin der Zugang zunächst einmal versperrt ist. Damit versuchen wir zu erzwingen, dass das Spiel ist im späteren Stadium konkreter wird.

