\section{Task 2}
Wir haben die folgenden Karten entworfen:

\subsection{2-Spieler-Karte}
Die zwei-Spieler-Karte wird von uns auch liebevoll \textit{Transversi} genannt. Sie ist eine Abwandlung der Standard Reversi Karte, wobei alle Felder ausschließlich über Transitionen verbunden sind. Die Motivation dahinter ist die, dass wir vor der Implementation von \textit{Blöcken} bei der Berechnung der Mobilität keine Transitionen beachtet haben. Wir hoffen, dass andere Gruppen dies auch nicht machen und wir mit unseren \textit{Blöcken} so einen Vorteil auf dieser Karte erhalten.

\subsection{4-Spieler-Karte}
Diese Karte trägt den Spitznamen \textit{Trophy of Loosers} und ist ein kleiner Hohn gegenüber den anderen Gruppen, die uns nie eine ernsthafte Bedrohung waren. Wir hoffen allerdings auf ein paar interessante Matches in dem finalen Turnier. Außerdem haben wir in dieser Karte ein paar Fallen eingebaut: Es gibt ein unerreichbares Inversion Feld. Dies ist eine Antwort auf die \textit{erreichbaren Komponenten}, an denen wir arbeiten. Wir vermuten, dass andere Gruppen ermitteln, wie viele Inversionstones es gibt und entsprechend ihr verhalten irgendwie anpassen. Mit einem unerreichbaren Stein erhoffen wir uns, dass andere Gruppen nicht feststellen, dass dieser niemals ausgeführt wird und deshalb schlechter spielen, so wie wir, ohne die \textit{erreichbaren Komponenten}. Die Bonus Felder im oberen Teil der Karte dienen auch dem Ziel, andere Gruppen aus dem Konzept zu bringen.

\subsection{8-Spieler-Karte}
