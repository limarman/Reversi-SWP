\section{Task 2}
Der Alpha-Beta Algorithmus ist nur eine kleine Anpassung des Minimax-Algorithmus. Mit zwei zusätzlichen Parametern und einer Abfrage mehr pro Aufruf des Min- oder Max-Spielers. Dementsprechend ist die Logik und der Callgraph gleich. Wir haben bereits eine Schnittstelle für das Sortieren der Züge eingebaut, was für zukünftige Assignments wahrscheinlich eine Rolle spielen wird. An dieser Stelle wollen wir dies jedoch \textit{zunächst} nicht weiter thematisieren.

Um das \textit{Alpha-Beta-Pruning} deaktivieren zu können, haben wir einen Commandline Parser geschrieben. Um diesen zu nutzen, muss man nur eine \textit{CliOption} erstellen und diese dem Parser hinzufügen. Nach einem Aufruf der \textit{parse(String[] args}-Methode, kann man die angegebenen Werte der Optionen aus den Objekten der Optionen auslesen. Sollte eine Option nicht gesetzt sein und einen weiteren Parameter haben (z.B. -s servername), wird ein Default-Wert zurückgegeben. Wenn nicht alle, als \textit{mandatory} markierten Optionen richtig gesetzt werden, gibt der Parser \textit{false} zurück und das Programm kann beendet werden. Eine Hilfeseite wird automatisch ausgegeben.