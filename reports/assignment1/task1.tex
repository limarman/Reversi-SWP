\section{Exercise 1}
\subsection{Task 5 - Evaluationsfunktion}
Eine kurze Recherche über die klassische Variante des Spiels und einige Testspiele gegen eine Reversi-KI ergaben folgenden Vorschlag für eine Evaluationsfunktion:
\subsubsection{Building phase}
Entgegengesetzt dem eigentlichen Spielziel in der building phase, welches wäre am Ende mit den meisten Steinen dazustehen, ergab sich aus empirischen Beobachtungen, dass die Anzahl der Steine am Anfang und in der Mitte des Spieles (fast) keinerlei Rolle spielt. Erst gegen Ende des Spieles wird dieser Faktor entscheidend.\\
Das ist der Grund, warum sich unsere Bewertungsfunktion unterteilt in Mittelspiel und  Endspiel. Im Mittelspiel werden die Anzahl der Steine nicht betrachtet, wohingegen sie am Ende einen sehr großen Teil der Bewertungsfunktion ausmachen.\\
Im \textbf{Mittelspiel} wird der Positionswert anhand folgender Kriterien beurteilt:\\

\underline{\textit{allgemeines Positionsspiel:}}\\
In diesem Teil werden hauptsächlich Aspekte der klassischen Variante von Reversi betrachtet und die zusätzlichen Möglichkeiten dem Abschnitt "spezielles Positionsspiel" überlassen.\\
Einen besonders hohen Wert haben sogenannte \textit{stabile Felder}, welche sich dadurch auszeichnen, dass sie von einem Gegner nicht mehr auf klassischem Wege übernommen werden können. Daran angrenzende sogenannte \textit{schwache Nachbarn}, sollten hingegen nicht besetzt werden, da sie den Gegnern gerade die Möglichkeit geben, die \textit{stabilen Felder} zu ergattern. Dementsprechend wird die Kontrolle eines \textit{stabilen Feldes} belohnt mit $PosVal_g$ += $5$ und die Kontrolle eines \textit{schwachen Nachbarn} bestraft mit $PosVal_g$ -= $5$. Man beachte jedoch folgendes: Wenn das stabile Feld einmal eingenommen ist, ist der Kampf um das Feld beendet und der \textit{schwache Nachbar} verliert seinen Status als schwach. Ist das \textit{stabile Feld} von einem selbst belegt, wird der \textit{schwache Nachbar} sogar zu einem guten Feld, da er zusätzliche Stabilität bekommt. Belohnung mit $PosVal_G$ += $2$.\\
Die Belegung \textit{stabilen Felder} ist aufgrund der Existenz von override stones jedoch nicht endgültig. Weswegen der allgemeine Positionswert skaliert wird mit der Anzahl der gegnerischen Overrides und verbleibenden Bonusfelder. Wir erhalten als allgemeinen Stellungswert schließlich:
$$rPosVal_G := \dfrac{PosVal_G}{\#OppOverrides + 0.5*\#Bonusfelder + 1}$$\\

\underline{\textit{spezielles Positionsspiel:}}\\
Hier wird versucht die speziellen Aspekte der Spielvariante zu bewerten.
Zunächst gibt es analog zu dem allgemeinen Positionsspiel auch hier \textit{schwache Nachbarn}, jedoch von Spezialfeldern. Einen sehr großen Wert haben Choice Felder. Weswegen die Besetzung angrenzender Felder abgestraft wird mit $PosVal_S$ -= $10$. Einen etwas weniger wichtigen Wert haben Bonusfelder, demnach wird die Strafe geringer: $PosVal_S$ -= $3$. Auch hier ist zu beachten, dass nach Einnahme der Spezialfelder, die Nachbarn wieder zu gewöhnlichen Feldern werden. Außerdem ist ein \textit{stabiles Feld} auch nie ein \textit{schwacher Nachbar}.
Da expansion stones keine schwerwiegende Spielveränderung mit sich führen und die Belegung von Inversionsfelder sowohl 'gut' als auch 'schlecht' sein können, werden diese der Einfachheit halber als normale Felder gehandhabt. Der letzte Punkt ist die eigene Anzahl der override stones. Diese sind ein sehr großer Vorteil, da sie einem Mobilität sichern (siehe Abschnitt 'Mobilität') und hervorragende Chancen das Spiel zum Ende hin zu seinem Vorteil zu drehen. Diese werden also großzügig belohnt mit $PosVal_S$ += $10 * \#EigeneOverrides$.\\

\underline{\textit{Mobilität:}}\\
Die Mobilität ist ein entscheidender Faktor in dem Spiel, da man bei Auslauf seiner Zugmöglichkeiten die kostbaren override stones verwenden oder sogar aussetzen muss. Außerdem hat man bei vielen Möglichkeiten eine gute Chance, dass eine besonders gute Variante dabei ist und kann nicht in einen ungünstigen forcierten Ablauf hineingezwängt werden. Man behält demnach eine gewisse Initiative. Wir bezeichnen Züge die keinen override stone kosten als \textit{kostenlose Züge}. Damit bekommt die Mobilität folgende Formel:
$$PosVal_M := -15 + 2*\#kostenloseZüge$$
Also bei 7-8 möglichen Zügen wäre der Break-Even-Punkt erreicht, was bei einem potentiell recht großen Feld wahrscheinlich noch zu klein gewählt ist. Außerdem sind wir uns noch nicht ganz sicher, ob der lineare Zusammenhang gut gewählt ist. Das muss noch zu sammelnde Spielerfahrung zeigen.
Insgesamt ergibt sich als Bewertungsfunktion im Mittelspiel:
$$PosVal := rPosVal_G + PosVal_S + PosVal_M$$
Im \textbf{Endspiel} vernachlässigen wir das Positionsspiel und konzentrieren uns auf den Punkt Mobilität und Feldkontrolle.\\
Wir setzen:$$PosVal := \dfrac{\#kostenloseZüge}{\#freieFelder} * 30 + \dfrac{ProzentKontrolliert}{100/\#Spieler} * 20$$
Die Skalierung der kostenlosen Züge kommt daher, dass natürlicherweise zum Ende der Bauphase hin die maximalen Möglichkeiten geringer werden. Die Skalierung des kontrollierten Spielfeldanteils, kommt daher, dass z.B. 12,5\% (=1/8) bei 8 Spielern und 50\% (=1/2) bei 2 Spielern gleichwertig ist und dementsprechend auch gleich viel in die Funktion einfließen soll.\\
Um einen sanfteren Übergang von Mittelspiel zu Endspiel zu schaffen fließen abhängig von der Anzahl an verbleibenden Spielzügen die beiden Bewertungsfunktionen unterschiedlich zusammen.
\begin{enumerate}
\item[\textbf{-}] >3 verbleibende Spielrunden -> $0.0*Endgame + 1.0*Middlegame$
\item[\textbf{-}] 3 verbleibende Spielrunden -> $0.1*Endgame + 0.9*Middlegame$
\item[\textbf{-}] 2 verbleibende Spielrunden -> $0.5*Endgame + 0.5*Middlegame$
\item[\textbf{-}] 1 verbleibende Spielrunde -> $1.0*Endgame + 0.0*Middlegame$
\end{enumerate}

\subsubsection{Bombing phase}