\section{Exercise 1}
Eine Beschreibung der Maps und eine Erörterung von möglichen Gewinnstrategien auf den jeweiligen Maps.
\subsection{Map1}
Map1 ist eine ziemlich simple Map, bei der es mehrere Löcher gibt, die entweder nur senkrecht und waagerecht oder auch diagonal die an den Löchern angrenzenden Tiles verbindet. 
Da es schon zu Beginn mehrere Anfangsmöglichkeiten gibt, ist es von Interesse wie die Spieler ihre ersten Züge realisieren. 
Wie wir in Task 5 erörtern ist die Mobilität von grosser Bedeutung. Deshalb wird es wichtig sein zu versuchen die Bereiche mit den Löchern wo nur senkrechte und waagerechte Transitionen existieren zu vermeiden, und somit ein Mobilitätsverlust zu vermeiden.
\subsubsection{Map2}
Bei Map2 ist die Idee mehrere seperate Reversi-Spielfelder in einem Spiel unterzubringen, die allesamt minimal verschieden sind und auch jeweils uneinnehmbare Felder an den Ecken besitzen.
Es ist von Interesse zu beobachten, welche Spieler welche Spielfelder prioritisieren und zu schauen ob die Spieler ein Spielfeld zuerst ausspielen oder zwischenzeitlich auch auf andere Spielfelder ausweichen.
Es ist davon auszugehen, dass auch wenn am Anfang die Möglichkeit besteht auf seperaten Feldern zu starten, dass im Laufe des Spiels mindestens zwei Spieler um ein Feld konkurrieren werden, da sonst für einen Spieler bei einem Feld ohne Konkurrenz schnell die Bewegungsmöglichkeiten ausgehen werden.
Wie bereits erwähnt, gibt es in jedem einzelnen Spielfeld uneinnehmbare Felder, die bei guten Gewinnstrategien von allen Spielern prioritisiert werden sollten, wenn sie einnehmbar sind. 
Man sollte deshalb auch versuchen es zu vermeiden, durch einen Zug einem anderen Spieler die Möglichkeit zu eröffnen diese Felder einzunehmen.
Ausserdem gibt es 3 Felder die grösser als die anderen sind und mehr Spezialfelder besitzen. Diese Felder sollten priorisiert werden, da man dort höhere Mobilität besitzt und mehr Felder einnehmen kann.
\subsubsection{Map3}
Bei dieser Map ist die Startmobilität sehr begrenzt und die Map besitzt zunächst auch keine Besonderheiten. Von daher ist davon auszugehen, dass die Partie anfangs wie ein normales Reversi Spiel verlaufen wird.
Jedoch gibt es dann aussenrum ein Rechteck bestehend aus Holes, der nur senkrechte, an den Ecken diagonale und keine waagerechte Transitionen bietet. Dies bedeutet, dass die Felder die an den Rechteck links und rechts angrenzen geringere Mobilität als andere Felder bieten.
Von daher sollte man als Spieler versuchen so gut wie möglich in senkrechter Richtung Felder einzunehmen um ein Mobilitätsvorteil zu erhalten und somit besseren Zugriff zu den ausserhalb liegenden Feldern zu bekommen.